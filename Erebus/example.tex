\documentclass[12pt,aspectratio=169]{beamer}
\usepackage[english]{babel}
% \usepackage[danish]{babel}  % Disse giver danske specialtegn
% \usepackage{ucs}            % Unicode
% \usepackage[T1]{fontenc}
\usepackage{venturis}       % skrifttype

\usetheme[]{Erebus}
\title[The Erebus Theme]{The Erebus Theme}
\subtitle[]{A further Development on the Hest Beamer Theme}
\author[Me!]{Stian L. Lybech}
\institute[AAU]{Aalborg Universitet}
\date{25/6-2019}

% Use XeLaTex - alway
% Compile this document twice - always
\begin{document}
\begin{frame}[plain,noframenumbering] 
  \maketitle
\end{frame}

% The usual way to insert the ToC
% \begin{frame}{Indhold}{}
%   \tableofcontents
% \end{frame}

% An alternative way to create the ToC in two columns
\begin{frame}
\frametitle{Indhold}
\begin{columns}[t]
  \begin{column}{.4\textwidth}
    \tableofcontents[sections={1-2}]
  \end{column}
  \begin{column}{.6\textwidth}
    \tableofcontents[sections={3-4}]
  \end{column}
\end{columns}
\end{frame}

%Section og subsection er kun relevant ift. TOC
\section{Some section}
\subsection{A subtitle}
\begin{frame}{Some section}{A subtitle}
Here is some text
  \begin{shadeblock}{Here is a shadeblock}
    \begin{itemize}
      \item An item
      \item Another item
    \end{itemize}
  \end{shadeblock}
\end{frame}

\begin{frame}
This is just another frame, which I have added to demonstrate the `clock' effect of the rotating slide counter in the upper right corner.
\end{frame}

% Use this command to insert a slide to signal a new topic in the presentation
\nextsection{Behind The Name}
\begin{frame}
\frametitle{The Meaning Behind The Name}
\framesubtitle{The Erebus Beamer Theme}
I created this theme for a presentation related to a project named ``Erebus,'' hence the name of the theme.
\end{frame}

\nextsection{End of Presentation}
\end{document}
