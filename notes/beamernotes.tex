\defbeamertemplate = definer en template
\setbeamertemplate = ANVEND en (allerede defineret) template
\usebeamertemplate = ??? Hvad er forskellen på den og setbeamertemplate?

Generelt om modes: 
1: \mode<presentation>{ ... } indsætter kun { ... }, når vi er i presentation mode.
2: \mode<presentation> vil "gobble" ALT efter denne linje, og frem til den næste \mode linje, når vi IKKE er i presentation mode.
3: \mode* The effect of this mode is to ignore all text outside frames in the presentation modes. In article mode it has no effect.

Man vælger en mode til sit dokument som en class-option til beamer

Mode-hierarki:
         <all> -> <article>       Article-mode: Til at opsætte tekst (fx send teksten videre til article.cls)
                | <presentation>  Den overordnede mode til slides

<presentation> -> <beamer>        Default mode
                | <second>        Noget med hvis man vil have opsat slides for to skærme
                | <handout>       Hvis man vil have lavet fx 4 slides pr A4-ark, til uddeling
                | <trans>         Til at lave transparenter (næppe noget, man skal bruge mere)

Fra manualen, s. 214:
Note: When a TEX file ends, TEX must not be in the gobbling state. 
Switch this state off using \mode on one line and <all> on the next.
Derfor vil man ofte se dette i slutningen af en beamer-fil eller en beamertheme<noget>.sty fil:
\mode
<all>


Overlay-aware commands:
\newcommand<>{\FLASH}[1]{\textcolor#2{red}{#1}}

Kommandoen er defineret med <> og med een parameter.
Der er altid een parameter mere, end det nummer man angiver i [n].
Det er overlay-specifikationen: I dette tilfælde #2.
Se Beamer-manualen s. 87

Effekten er nøjagtigt den samme som \alert, der også kan bruges med overlay specifiationer, fx \alert<2->{Hest}
Farven kan ændres som en del af beamer-temaet med
\setbeamercolor{alerted text}{fg=yellow}

Nemmeste måde at lave pile der peger på tekst:
 \def\tikzmark(#1){%
 \tikz[remember picture]{%
 \coordinate(#1);}%
 }

% Fix token not allowed in pdf string mht. græske bogstaver, 
% hvis de bruges i overskrifter
\pdfstringdefDisableCommands{%
  \def\rho{rho}
  \def\pi{pi}
  \def\lambda{lambda}
}

% Indsæt noget, så det ligner et stykke papir:
\tikzset{
  paper/.style = {shape = rectangle, draw = black, inner sep = 0pt, anchor = south east, drop shadow}
}

\def\includepaper[#1] at (#2) #3;{ \tikz[remember picture, overlay] \node[paper, #1] at (#2) {\includegraphics[width = 4cm]{#3}}; }

% Eksempelvis to ark oven på hinanden:
% Tag et skærmprint af PDF'ens forside, og inkluder det:
\onslide<1->{ \includepaper[rotate = -5] at ([xshift = -1.5cm, yshift = 1cm]current page.south east) {gorlaconference.png}; }
\onslide<2->{ \includepaper[rotate = 10] at ([xshift = -0.5cm, yshift = 1.5cm]current page.south east) {gorlapaper.png}; }

% Automatic uncovering
\begin{itemize}[<+->]
  \item Bla
  \item Blabla
\end{itemize}

% To start uncovering from slide 2 instead of slide 1
\begin{itemize}[<+(1)->]
  \item Bla
  \item Blabla
\end{itemize}


