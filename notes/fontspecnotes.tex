% Sådan indlæser man forskellige font-filer incl. filer man har lokalt i mappen

\usepackage{fontspec}

% /usr/share/texmf-dist/fonts/opentype/public/tex-gyre/
\setmainfont{texgyreschola}[
  Extension      = .otf,
  UprightFont    = *-regular,
  ItalicFont     = *-italic,
  BoldFont       = *-bold,
  BoldItalicFont = *-bolditalic,
  Numbers        = OldStyle
]

\setmathrm{texgyreschola-math.otf}
\setmathsf{texgyreschola-math.otf}
\setmathtt{texgyreschola-math.otf}
\setboldmathrm{texgyreschola-math.otf}

% Definer en ny fontkommando ud fra en font fil
\newfontfamily{\phonemefont}{CharisSILCompact}[
  Path           = fonts/,
  Extension      = .ttf,
  UprightFont    = *-R,
  ItalicFont     = *-I,
  BoldFont       = *-B,
  BoldItalicFont = *-BI
]

% /usr/share/texmf-dist/fonts/opentype/public/qualitype/
\newfontfamily{\blackletterfont}{QTFraktur}[Extension = .otf]
\newfontfamily{\handwritingfont}{QTHandwriting}[Extension = .otf]

% Linux libertine har en lidt underlig navngivning
% Biolinum er deres sans-font
% LinBiolinum_K.otf     % Keyboard font (til at lave noget der ligner keyboard taster)
% LinBiolinum_R.otf     % Regular
% LinBiolinum_RB.otf    % Regular Bold
% LinBiolinum_RBO.otf   % Regular Bold "Oblique to fake italics" (??)
% LinBiolinum_RI.otf    % Regular Italics
% 
% Libertine er deres serif-font
% LinLibertine_M.otf    % Monospaced
% LinLibertine_MB.otf   % Monospaced Bold
% LinLibertine_MBO.otf  % Monospaced Bold Oldstyle
% LinLibertine_MO.otf   % Monospaced Oblique
% 
% LinLibertine_R.otf    % Regular
% LinLibertine_RB.otf   % Regular Bold
% LinLibertine_RBI.otf  % Regular Bold Italics
% LinLibertine_RI.otf   % Regular Italics

% Disse kan bruges i stedet for den almindelige Regular og Regular Italics
% LinLibertine_RZ.otf   % Regular SemiBold
% LinLibertine_RZI.otf  % Regular SemiBold Italics
% 
% LinLibertine_DR.otf   % Display Regular (jeg aner ikke, hvad den er til)
% LinLibertine_I.otf    % Initials? (tror jeg)

\setsansfont{LinBiolinum}[
  Extension      = .otf,
  UprightFont    = *_R,
  ItalicFont     = *_RI,
  BoldFont       = *_RB
]

\setmainfont{LinLibertine}[
  Extension      = .otf,
  UprightFont    = *_R,
  ItalicFont     = *_RI,
  BoldFont       = *_RB,
  BoldItalicFont = *_RBI,
  Numbers        = OldStyle
]

\setmonofont{LinLibertine}[
  Extension      = .otf,
  UprightFont    = *_M,
  ItalicFont     = *_MO,
  BoldFont       = *_MB,
  BoldItalicFont = *_MBO
]

\setmathrm{LibertinusMath-Regular.otf}
\setmathsf{LibertinusMath-Regular.otf}
\setmathtt{LibertinusMath-Regular.otf}
\setboldmathrm{LibertinusMath-Regular.otf}
