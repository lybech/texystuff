LATEX AND TEX COMMANDS

\newcommand og \newcommand*
You can say \long\def\foo{...}. This defines the \foo command to be \long, said otherwise, \par tokens are allowed inside the argument. In the case of \newcommand\foo{...} the result is a \long command, and \newcommand*\foo{...} produces a non-\long one.

\AtEndDocument

\newcommand{\mycommandStar}[2]{starred version}
\newcommand{\mycommandNoStar}[1]{normal version}
\newcommand{\mycommand}{\@ifstar
                     \mycommandStar%
                     \mycommandNoStar%
}

##1:
Hvis jeg definerer en kommando (B) inde i definitionen af en anden kommando (A), skal jeg bruge ##1 til Bs interne parameter:
\newcommand{\mycommand}{%
\newcommand{\testkommando}[2]{Dette ##1 er en test ##2}
}
\mycommmand opretter \testkommando med to parametre.

The \providecommand command is identical to the \newcommand command if a command with this name does not exist; if it does already exist, the \providecommand does nothing and the old definition remains in effect.

%\the...
Hvis counteren HEST findes, vil kommandoen \theHEST printe dens værdi.

DEF og EDEF:
\def\hest#1#2{Hej #1 og #2}
\edef\ehest#1#2{\value{blah} + #1 - #2}
In \def, values are not expanded, until the macro is called
In \edef, values are expanded at definition-time

IFS:
\newif\if@hest\@hestfalse
\newcommand*{\hest}{\@hesttrue}
\if@hest\dosomething\else\dosomethingelse\fi

Først definerer jeg min bool-variabel "@hest" og sætter dens default værdi til false.
Dernæst definerer jeg en kommando, der sætter den til true.
Endelig kan jeg så bruge \if@hest til at gøre \dosomething, hvis \@hest er true, og noget andet, hvis den er false.

PACKAGE OPTIONS:
\newif\if@hest\@hestfalse
\def\havehest{\@hesttrue}
\DeclareOption{hest}{\havegrl}
\ProcessOptions %Options are processed here. So the IF is not set to true until AFTER this point

\if@hest
  \typeout{PRUUUUH!}
\else
  \typeout{Ingen hest :(}
\fi
