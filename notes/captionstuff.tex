\section{Errors in caption}
Everything in a caption is a `moving argument'.
Many commands will therefore fail with bizarre errors if used in captions, unless they are prefixed with \protect or declared as robust.
Here are two ways to insert coloured squares in the text:
Either declare a robust command:

\DeclareRobustCommand{\colorsquare}[1]{\tikz\node[shape = rectangle, fill = #1] {};}

Or prefix every command with \protect; i.e. both \tikz and \node:

\begin{figure}
...
\caption{The tasks marked in green \colorsquare{green!75!black} pertain to X. 
The yellow \protect\tikz\protect\node[shape = rectangle, fill = yellow!75!black] {}; pertain to Y.}
\end{figure}


\section{Styling captions in memoir}
How to typeset the name of the type of the figure: e.g. `figure' or `table'
\captionnamefont{\small\bfseries}

How to typeset the rest of the text:
\captiontitlefont{\small}

Center the caption, if it is too short to fill a whole line.
Ellers brug lige marginer (som standard):
\captionstyle[\centering]{} % evt. \captionstyle[\centering]{\raggedright} hvis vi ikke vil have en lige højremargin

Bredden af en caption:
%\captionwidth{\linewidth}\changecaptionwidth
%\captionwidth{.9\linewidth}\changecaptionwidth

Størrelsen på mellemrummet mellem figuren og caption, og mellem caption of teksten på siden.
\setlength{\abovecaptionskip}{0pt}
\setlength{\belowcaptionskip}{0pt}
