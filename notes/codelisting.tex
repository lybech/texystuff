\documentclass[testmain.tex]{subfiles}

% Pakke til code listing
\usepackage{listings} 

% Opsætning af farver til code listing
\definecolor{strings}{rgb}{0,0.5,0}
\definecolor{emphs}{rgb}{0.64,0.08,0.08}
\definecolor{comments}{rgb}{0.17,0.57,0.68}
\colorlet{keywords}{blue!50!cyan}

% Definition af den style, vi ønsker at bruge
\lstdefinestyle{mycodestyle}{
  language=C,
  captionpos=b,
  numbers=left,
  numberstyle=\tiny,
  frame=lines,
  showspaces=false,
  showtabs=false,
  breaklines=true,
  showstringspaces=false,
  breakatwhitespace=true,
  emph={malloc, calloc, realloc, free, memset}, % tilføj selv andre ting fra stdlib, som I vil have fremhævet
  emphstyle={\rmfamily\bfseries\color{emphs}},
  commentstyle=\color{comments},
  morekeywords={partial, var, value, get, set},
  keywordstyle={\bfseries\color{keywords}},
  stringstyle=\color{strings},
  basicstyle=\ttfamily\small,
  escapechar=@
}


\begin{document}
\chapter{Dette er en test}

\begin{figure}
% Sæt style, caption og label (til brug for ref) 
\begin{lstlisting}[style=mycodestyle, caption={Udpakning af kromosomet}, label=lst:test]
/* m = antal e-mails i alt (kromosomets sande laengde) */
int * unpack_chromosome(int *w, int m) {
  int i = 0,
      j = w[0],   /* Antallet af elementer incl w[0] */
      *wm = NULL; /* Det udpakkede kromosom */
      
  /* alloker plads og saet alle elementer til nul */
  wm = (int *) malloc(m * sizeof(int)); @\label{line:malloc}@
  memset(wm, 0, m * sizeof(int));
  
  /* "udpak" kromosomet */
  for (i = 1; i < j; i++)
    wm[w[i]] = 1;

  return wm;
}
\end{lstlisting}
\end{figure}

% Her er et eksempel på brug af ref til at indsætte det korrekte linjenummer
På linje~\ref{line:malloc} i listing~\ref{lst:test} ser vi et kald til \texttt{malloc}.
Det er pænt.

\end{document}
